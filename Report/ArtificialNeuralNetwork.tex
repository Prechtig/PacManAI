% !TEX root = Report.tex

\section{Artificial Neural Network}
\subsection{Description}
An ANN consists of multiple layers of neurons. In a traditional ANN, each layer of neurons is fully connected to every neuron in the adjacent layers. A connection between two neurons has a weight and a neuron has a bias. 

The first layer is the input neurons, and the last layer is the output neurons. In addition to the input and output layers, there is the optional hidden layer, which consists of any number of layers.

The output of input neurons is calculated as $ I_i + bias_i $, where $I_i$ is the input and $bias_i$ is the bias of the \textit{i'th} input neuron.

The output of neurons in the subsequent layers are calculated as $ \left( \displaystyle \sum_{i = 0}^{n} O_i * W_i \right) + bias $, where $O_i$ is the output from neuron $i$ in the previous layer, $W_i$ is the weight between neuron $i$ in the previous layer and the current neuron and $bias$ is the bias of the current neuron.

\subsection{Inputs to the Artificial Neural Network}
\label{subsub:ann_inputs}
The ANN developed for playing Ms. Pac-Man takes ten inputs: distances to all four ghosts, whether each ghost is edible, distance to nearest power pill and distance to the nearest pill.

\subsection{Outputs from the Artificial Neural Network}
\label{subsub:ann_outputs}
In this particular ANN there is only one output node. This node is interpreted as how good a move is.

Every time the Ms. Pac-Man framework requests a move, the list of possible moves are. For each possible move a run through the ANN is made, to determine how good that move is. The move with the highest output is then chosen as the move to do.

Another approach that could have been chosen, was to have four output nodes, each representing one of the four unique moves.

\subsection{Teaching an Artificial Neural Network}
There are multiple ways to learn an ANN how to do best. The four major ways are: Supervised learning, unsupervised learning, reinforcement learning and evolutionary learning.

\subsubsection{Supervised learning}
Supervised learning requires example data, where the input and output is known. By minimising the difference between the actual output and the expected output you can learn the network what to do.

\subsubsection{Unsupervised learning}
Unsupervised learning does not require example data. A cost function is chosen and the goal is to minimise the cost.

\subsubsection{Reinforcement learning}
Instead of example data reinforcement learning observes someone or something else playing, using that as the reference.

\subsubsection{Evolutionary learning}
A completely different approach for teaching an ANN is to use an evolutionary approach. The idea comes from biology and the ongoing evolution of every species. By using evolution it is possible to evolve ANN that can solve tasks without learning it what to do, for instance by making small mutations to networks and evolving the best ones.

\subsection{Teaching an ANN to play Ms. Pac-Man}
The chosen teaching paradigm is evolution since it fits the jobs the most. There is no training data neither a cost function. Using reinforcement learning imposes an upper bound of ones own skill at playing Ms. Pac-Man.

\subsubsection{Algorithm parameters}
In order to teach the ANN to play Ms. Pac-Man an evolution framework was developed. Both the ANN and the evolution framework has multiple parameters to play with in order to achieve the best performing AI. \\

\noindent The important parameters include the following:

\begin{itemize}
	\item Network
	\begin{itemize}
		\item Number of input neurons
		\item Number of output neurons
		\item Number of neurons in the hidden layer
		\item Initial weight of connections
		\item Initial bias of neurons
	\end{itemize}
	
	\item Evolution
	\begin{itemize}
		\item Evaluations per network
		\item Networks in a generation
		\item Number of elitists
		\item Number of parents
		\item Chance of mutation
		\item Intensity of mutation
	\end{itemize}
\end{itemize}

\noindent Several runs of the evolution framework has been made with different kind of parameters in order to find the best performing network.

The number of input neurons defines how many inputs that are given to the network and the number of output neurons defines how many outputs the network will produce. The number of neurons in the hidden layer defines how many neurons there are in the hidden layer. The evolution framework does support multiple hidden layers, but is cumbersome to use at the moment. This decision was made to limit the amount of parameters and thereby limiting the search space.

Initial weight of connections is used to set the initial weight of the connections in the first network. Initial bias of neurons is the same as the aforementioned. A different approach could be to just randomise every weight and bias of the network. I chose the fixed value approach because it is slightly simpler to implement, and the value will change over time anyway.

Evaluations per network is how many games are played with a given network. Since Ms. Pac-Man is not deterministic, one can use a high number of evaluations to minimise the noise created from the indeterministic behaviour.

The number of elitists are the survivors of each generation. These are the networks that will be transferred to the next generation without being mutated.

The number of parents is the amount n best scoring networks that will be mutated in order to produce the remaining population.

Chance and intensity of mutation is how likely	 a given weight or neuron is to mutate, and how intense that mutation is.

\subsection{Experiments}
In order to develop the best network I ran the evolution framework several times adjusting different parameters each time. A small sample of these experiments are shown in the following table. For all experiments shown, the chance to mutate was 0.1 and the intensity of mutation was 0.2. The size of every generation was 100 and each network was evaluated 25 times.

% Column definition used to make fixed width right aligned columns
\newcolumntype{R}[1]{>{\raggedleft\arraybackslash}p{#1\columnwidth}}

\noindent \\
\begin{tabular}{ | r | r | r | r | }
	\hline
	Hidden neurons & Parents & Avg. score & Best avg. score \\ \hline
	0  & 2  & 2080 & 2632 \\ \hline
	0  & 10 & 2345 & 2868 \\ \hline
	7  & 2  & 1634 & 3458 \\ \hline
	7  & 10 & 2078 & 2799 \\ \hline
	15 & 2  & 2208 & 2696 \\ \hline
	15 & 10 & 2273 & 2635 \\
	\hline
\end{tabular}

\noindent \\
The best performing network i found uses ten inputs as described in section~\ref{subsub:ann_inputs} and one output as described in section~\ref{subsub:ann_outputs}. The number of neurons in the hidden layer are seven.