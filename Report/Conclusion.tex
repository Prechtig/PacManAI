% !TEX root = Report.tex

\section{Conclusion}
It is possible to develop AI that is capable of playing Ms. Pac-Man at a reasonable level.

The algorithm that performed the best was by far the Behaviour Tree. The BT was able to score an average of around 6300 which is somewhere in the range of my own capabilities.

The worst performing algorithm was the Monte Carlo Tree Search. It was able to get an average score of around 1100 which is really bad. After having observed the MCTS controller playing it is quite obvious that there is a bug somewhere. Sometimes when Pac-Man has a ghost just millimeters behind it, the controller decides to switch direction and go straight into the ghost - even though continuing would not result in dying. This is a huge indication that the algorithm is bugged at the moment. Unfortunately I did not have time to correct this.

The performance of the Artificial Neural Network was not impressive, but it did all right. The problem with ANN is that it takes a really long time to find a network that is suitable. Due to time constraints the implemented evolution framework did not feature evolution of the topology. This means that the network would not change in terms of the amount of neurons and layers. I believe it would have been possible to get better results if the evolution framework could change the topology of the network. Furthermore trying different amounts of layers in the hidden layer could also yield better results, but this would be gained from changing the topology.