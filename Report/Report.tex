%% bare_jrnl_compsoc.tex
%% V1.4a
%% 2014/09/17
%% by Michael Shell
%% See:
%% http://www.michaelshell.org/
%% for current contact information.
%%
%% This is a skeleton file demonstrating the use of IEEEtran.cls
%% (requires IEEEtran.cls version 1.8a or later) with an IEEE
%% Computer Society journal paper.
%%
%% Support sites:
%% http://www.michaelshell.org/tex/ieeetran/
%% http://www.ctan.org/tex-archive/macros/latex/contrib/IEEEtran/
%% and
%% http://www.ieee.org/

%%*************************************************************************
%% Legal Notice:
%% This code is offered as-is without any warranty either expressed or
%% implied; without even the implied warranty of MERCHANTABILITY or
%% FITNESS FOR A PARTICULAR PURPOSE! 
%% User assumes all risk.
%% In no event shall IEEE or any contributor to this code be liable for
%% any damages or losses, including, but not limited to, incidental,
%% consequential, or any other damages, resulting from the use or misuse
%% of any information contained here.
%%
%% All comments are the opinions of their respective authors and are not
%% necessarily endorsed by the IEEE.
%%
%% This work is distributed under the LaTeX Project Public License (LPPL)
%% ( http://www.latex-project.org/ ) version 1.3, and may be freely used,
%% distributed and modified. A copy of the LPPL, version 1.3, is included
%% in the base LaTeX documentation of all distributions of LaTeX released
%% 2003/12/01 or later.
%% Retain all contribution notices and credits.
%% ** Modified files should be clearly indicated as such, including  **
%% ** renaming them and changing author support contact information. **
%%
%% File list of work: IEEEtran.cls, IEEEtran_HOWTO.pdf, bare_adv.tex,
%%                    bare_conf.tex, bare_jrnl.tex, bare_conf_compsoc.tex,
%%                    bare_jrnl_compsoc.tex, bare_jrnl_transmag.tex
%%*************************************************************************


% *** Authors should verify (and, if needed, correct) their LaTeX system  ***
% *** with the testflow diagnostic prior to trusting their LaTeX platform ***
% *** with production work. IEEE's font choices and paper sizes can       ***
% *** trigger bugs that do not appear when using other class files.       ***                          ***
% The testflow support page is at:
% http://www.michaelshell.org/tex/testflow/


\documentclass[10pt,journal,compsoc]{IEEEtran}


%% *** CUSTOM PACKAGES ***
\usepackage{array}
\usepackage{qtree}
\usepackage{tikz-qtree}
\usepackage{neuralnetwork}

% *** CITATION PACKAGES ***
%
\ifCLASSOPTIONcompsoc
  % IEEE Computer Society needs nocompress option
  % requires cite.sty v4.0 or later (November 2003)
  \usepackage[nocompress]{cite}
\else
  % normal IEEE
  \usepackage{cite}
\fi
% cite.sty was written by Donald Arseneau
% V1.6 and later of IEEEtran pre-defines the format of the cite.sty package
% \cite{} output to follow that of IEEE. Loading the cite package will
% result in citation numbers being automatically sorted and properly
% "compressed/ranged". e.g., [1], [9], [2], [7], [5], [6] without using
% cite.sty will become [1], [2], [5]--[7], [9] using cite.sty. cite.sty's
% \cite will automatically add leading space, if needed. Use cite.sty's
% noadjust option (cite.sty V3.8 and later) if you want to turn this off
% such as if a citation ever needs to be enclosed in parenthesis.
% cite.sty is already installed on most LaTeX systems. Be sure and use
% version 5.0 (2009-03-20) and later if using hyperref.sty.
% The latest version can be obtained at:
% http://www.ctan.org/tex-archive/macros/latex/contrib/cite/
% The documentation is contained in the cite.sty file itself.
%
% Note that some packages require special options to format as the Computer
% Society requires. In particular, Computer Society  papers do not use
% compressed citation ranges as is done in typical IEEE papers
% (e.g., [1]-[4]). Instead, they list every citation separately in order
% (e.g., [1], [2], [3], [4]). To get the latter we need to load the cite
% package with the nocompress option which is supported by cite.sty v4.0
% and later. Note also the use of a CLASSOPTION conditional provided by
% IEEEtran.cls V1.7 and later.





% *** GRAPHICS RELATED PACKAGES ***
%
\ifCLASSINFOpdf
  % \usepackage[pdftex]{graphicx}
  % declare the path(s) where your graphic files are
  % \graphicspath{{../pdf/}{../jpeg/}}
  % and their extensions so you won't have to specify these with
  % every instance of \includegraphics
  % \DeclareGraphicsExtensions{.pdf,.jpeg,.png}
\else
  % or other class option (dvipsone, dvipdf, if not using dvips). graphicx
  % will default to the driver specified in the system graphics.cfg if no
  % driver is specified.
  % \usepackage[dvips]{graphicx}
  % declare the path(s) where your graphic files are
  % \graphicspath{{../eps/}}
  % and their extensions so you won't have to specify these with
  % every instance of \includegraphics
  % \DeclareGraphicsExtensions{.eps}
\fi
% graphicx was written by David Carlisle and Sebastian Rahtz. It is
% required if you want graphics, photos, etc. graphicx.sty is already
% installed on most LaTeX systems. The latest version and documentation
% can be obtained at: 
% http://www.ctan.org/tex-archive/macros/latex/required/graphics/
% Another good source of documentation is "Using Imported Graphics in
% LaTeX2e" by Keith Reckdahl which can be found at:
% http://www.ctan.org/tex-archive/info/epslatex/
%
% latex, and pdflatex in dvi mode, support graphics in encapsulated
% postscript (.eps) format. pdflatex in pdf mode supports graphics
% in .pdf, .jpeg, .png and .mps (metapost) formats. Users should ensure
% that all non-photo figures use a vector format (.eps, .pdf, .mps) and
% not a bitmapped formats (.jpeg, .png). IEEE frowns on bitmapped formats
% which can result in "jaggedy"/blurry rendering of lines and letters as
% well as large increases in file sizes.
%
% You can find documentation about the pdfTeX application at:
% http://www.tug.org/applications/pdftex






% *** MATH PACKAGES ***
%
%\usepackage[cmex10]{amsmath}
% A popular package from the American Mathematical Society that provides
% many useful and powerful commands for dealing with mathematics. If using
% it, be sure to load this package with the cmex10 option to ensure that
% only type 1 fonts will utilized at all point sizes. Without this option,
% it is possible that some math symbols, particularly those within
% footnotes, will be rendered in bitmap form which will result in a
% document that can not be IEEE Xplore compliant!
%
% Also, note that the amsmath package sets \interdisplaylinepenalty to 10000
% thus preventing page breaks from occurring within multiline equations. Use:
%\interdisplaylinepenalty=2500
% after loading amsmath to restore such page breaks as IEEEtran.cls normally
% does. amsmath.sty is already installed on most LaTeX systems. The latest
% version and documentation can be obtained at:
% http://www.ctan.org/tex-archive/macros/latex/required/amslatex/math/





% *** SPECIALIZED LIST PACKAGES ***
%
%\usepackage{algorithmic}
% algorithmic.sty was written by Peter Williams and Rogerio Brito.
% This package provides an algorithmic environment fo describing algorithms.
% You can use the algorithmic environment in-text or within a figure
% environment to provide for a floating algorithm. Do NOT use the algorithm
% floating environment provided by algorithm.sty (by the same authors) or
% algorithm2e.sty (by Christophe Fiorio) as IEEE does not use dedicated
% algorithm float types and packages that provide these will not provide
% correct IEEE style captions. The latest version and documentation of
% algorithmic.sty can be obtained at:
% http://www.ctan.org/tex-archive/macros/latex/contrib/algorithms/
% There is also a support site at:
% http://algorithms.berlios.de/index.html
% Also of interest may be the (relatively newer and more customizable)
% algorithmicx.sty package by Szasz Janos:
% http://www.ctan.org/tex-archive/macros/latex/contrib/algorithmicx/




% *** ALIGNMENT PACKAGES ***
%
%\usepackage{array}
% Frank Mittelbach's and David Carlisle's array.sty patches and improves
% the standard LaTeX2e array and tabular environments to provide better
% appearance and additional user controls. As the default LaTeX2e table
% generation code is lacking to the point of almost being broken with
% respect to the quality of the end results, all users are strongly
% advised to use an enhanced (at the very least that provided by array.sty)
% set of table tools. array.sty is already installed on most systems. The
% latest version and documentation can be obtained at:
% http://www.ctan.org/tex-archive/macros/latex/required/tools/


% IEEEtran contains the IEEEeqnarray family of commands that can be used to
% generate multiline equations as well as matrices, tables, etc., of high
% quality.




% *** SUBFIGURE PACKAGES ***
%\ifCLASSOPTIONcompsoc
%  \usepackage[caption=false,font=footnotesize,labelfont=sf,textfont=sf]{subfig}
%\else
%  \usepackage[caption=false,font=footnotesize]{subfig}
%\fi
% subfig.sty, written by Steven Douglas Cochran, is the modern replacement
% for subfigure.sty, the latter of which is no longer maintained and is
% incompatible with some LaTeX packages including fixltx2e. However,
% subfig.sty requires and automatically loads Axel Sommerfeldt's caption.sty
% which will override IEEEtran.cls' handling of captions and this will result
% in non-IEEE style figure/table captions. To prevent this problem, be sure
% and invoke subfig.sty's "caption=false" package option (available since
% subfig.sty version 1.3, 2005/06/28) as this is will preserve IEEEtran.cls
% handling of captions.
% Note that the Computer Society format requires a sans serif font rather
% than the serif font used in traditional IEEE formatting and thus the need
% to invoke different subfig.sty package options depending on whether
% compsoc mode has been enabled.
%
% The latest version and documentation of subfig.sty can be obtained at:
% http://www.ctan.org/tex-archive/macros/latex/contrib/subfig/




% *** FLOAT PACKAGES ***
%
%\usepackage{fixltx2e}
% fixltx2e, the successor to the earlier fix2col.sty, was written by
% Frank Mittelbach and David Carlisle. This package corrects a few problems
% in the LaTeX2e kernel, the most notable of which is that in current
% LaTeX2e releases, the ordering of single and double column floats is not
% guaranteed to be preserved. Thus, an unpatched LaTeX2e can allow a
% single column figure to be placed prior to an earlier double column
% figure. The latest version and documentation can be found at:
% http://www.ctan.org/tex-archive/macros/latex/base/


%\usepackage{stfloats}
% stfloats.sty was written by Sigitas Tolusis. This package gives LaTeX2e
% the ability to do double column floats at the bottom of the page as well
% as the top. (e.g., "\begin{figure*}[!b]" is not normally possible in
% LaTeX2e). It also provides a command:
%\fnbelowfloat
% to enable the placement of footnotes below bottom floats (the standard
% LaTeX2e kernel puts them above bottom floats). This is an invasive package
% which rewrites many portions of the LaTeX2e float routines. It may not work
% with other packages that modify the LaTeX2e float routines. The latest
% version and documentation can be obtained at:
% http://www.ctan.org/tex-archive/macros/latex/contrib/sttools/
% Do not use the stfloats baselinefloat ability as IEEE does not allow
% \baselineskip to stretch. Authors submitting work to the IEEE should note
% that IEEE rarely uses double column equations and that authors should try
% to avoid such use. Do not be tempted to use the cuted.sty or midfloat.sty
% packages (also by Sigitas Tolusis) as IEEE does not format its papers in
% such ways.
% Do not attempt to use stfloats with fixltx2e as they are incompatible.
% Instead, use Morten Hogholm'a dblfloatfix which combines the features
% of both fixltx2e and stfloats:
%
% \usepackage{dblfloatfix}
% The latest version can be found at:
% http://www.ctan.org/tex-archive/macros/latex/contrib/dblfloatfix/




%\ifCLASSOPTIONcaptionsoff
%  \usepackage[nomarkers]{endfloat}
% \let\MYoriglatexcaption\caption
% \renewcommand{\caption}[2][\relax]{\MYoriglatexcaption[#2]{#2}}
%\fi
% endfloat.sty was written by James Darrell McCauley, Jeff Goldberg and 
% Axel Sommerfeldt. This package may be useful when used in conjunction with 
% IEEEtran.cls'  captionsoff option. Some IEEE journals/societies require that
% submissions have lists of figures/tables at the end of the paper and that
% figures/tables without any captions are placed on a page by themselves at
% the end of the document. If needed, the draftcls IEEEtran class option or
% \CLASSINPUTbaselinestretch interface can be used to increase the line
% spacing as well. Be sure and use the nomarkers option of endfloat to
% prevent endfloat from "marking" where the figures would have been placed
% in the text. The two hack lines of code above are a slight modification of
% that suggested by in the endfloat docs (section 8.4.1) to ensure that
% the full captions always appear in the list of figures/tables - even if
% the user used the short optional argument of \caption[]{}.
% IEEE papers do not typically make use of \caption[]'s optional argument,
% so this should not be an issue. A similar trick can be used to disable
% captions of packages such as subfig.sty that lack options to turn off
% the subcaptions:
% For subfig.sty:
% \let\MYorigsubfloat\subfloat
% \renewcommand{\subfloat}[2][\relax]{\MYorigsubfloat[]{#2}}
% However, the above trick will not work if both optional arguments of
% the \subfloat command are used. Furthermore, there needs to be a
% description of each subfigure *somewhere* and endfloat does not add
% subfigure captions to its list of figures. Thus, the best approach is to
% avoid the use of subfigure captions (many IEEE journals avoid them anyway)
% and instead reference/explain all the subfigures within the main caption.
% The latest version of endfloat.sty and its documentation can obtained at:
% http://www.ctan.org/tex-archive/macros/latex/contrib/endfloat/
%
% The IEEEtran \ifCLASSOPTIONcaptionsoff conditional can also be used
% later in the document, say, to conditionally put the References on a 
% page by themselves.




% *** PDF, URL AND HYPERLINK PACKAGES ***
%
\usepackage{url}
% url.sty was written by Donald Arseneau. It provides better support for
% handling and breaking URLs. url.sty is already installed on most LaTeX
% systems. The latest version and documentation can be obtained at:
% http://www.ctan.org/tex-archive/macros/latex/contrib/url/
% Basically, \url{my_url_here}.





% *** Do not adjust lengths that control margins, column widths, etc. ***
% *** Do not use packages that alter fonts (such as pslatex).         ***
% There should be no need to do such things with IEEEtran.cls V1.6 and later.
% (Unless specifically asked to do so by the journal or conference you plan
% to submit to, of course. )


% correct bad hyphenation here
\hyphenation{op-tical net-works semi-conduc-tor}


\begin{document}
%
% paper title
% Titles are generally capitalized except for words such as a, an, and, as,
% at, but, by, for, in, nor, of, on, or, the, to and up, which are usually
% not capitalized unless they are the first or last word of the title.
% Linebreaks \\ can be used within to get better formatting as desired.
% Do not put math or special symbols in the title.
\title{Implementing AI in Ms. Pac-Man}
%
%
% author names and IEEE memberships
% note positions of commas and nonbreaking spaces ( ~ ) LaTeX will not break
% a structure at a ~ so this keeps an author's name from being broken across
% two lines.
% use \thanks{} to gain access to the first footnote area
% a separate \thanks must be used for each paragraph as LaTeX2e's \thanks
% was not built to handle multiple paragraphs
%
%
%\IEEEcompsocitemizethanks is a special \thanks that produces the bulleted
% lists the Computer Society journals use for "first footnote" author
% affiliations. Use \IEEEcompsocthanksitem which works much like \item
% for each affiliation group. When not in compsoc mode,
% \IEEEcompsocitemizethanks becomes like \thanks and
% \IEEEcompsocthanksitem becomes a line break with idention. This
% facilitates dual compilation, although admittedly the differences in the
% desired content of \author between the different types of papers makes a
% one-size-fits-all approach a daunting prospect. For instance, compsoc 
% journal papers have the author affiliations above the "Manuscript
% received ..."  text while in non-compsoc journals this is reversed. Sigh.

\author{Andreas~Precht~Poulsen, \textit{appo@itu.dk}, October 9, 2015}

% note the % following the last \IEEEmembership and also \thanks - 
% these prevent an unwanted space from occurring between the last author name
% and the end of the author line. i.e., if you had this:
% 
% \author{....lastname \thanks{...} \thanks{...} }
%                     ^------------^------------^----Do not want these spaces!
%
% a space would be appended to the last name and could cause every name on that
% line to be shifted left slightly. This is one of those "LaTeX things". For
% instance, "\textbf{A} \textbf{B}" will typeset as "A B" not "AB". To get
% "AB" then you have to do: "\textbf{A}\textbf{B}"
% \thanks is no different in this regard, so shield the last } of each \thanks
% that ends a line with a % and do not let a space in before the next \thanks.
% Spaces after \IEEEmembership other than the last one are OK (and needed) as
% you are supposed to have spaces between the names. For what it is worth,
% this is a minor point as most people would not even notice if the said evil
% space somehow managed to creep in.



% The paper headers
\markboth{Journal, October~2015}%
{Shell \MakeLowercase{\textit{et al.}}: Implementing AI in Ms. Pac-Man}
% The only time the second header will appear is for the odd numbered pages
% after the title page when using the twoside option.
% 
% *** Note that you probably will NOT want to include the author's ***
% *** name in the headers of peer review papers.                   ***
% You can use \ifCLASSOPTIONpeerreview for conditional compilation here if
% you desire.



% The publisher's ID mark at the bottom of the page is less important with
% Computer Society journal papers as those publications place the marks
% outside of the main text columns and, therefore, unlike regular IEEE
% journals, the available text space is not reduced by their presence.
% If you want to put a publisher's ID mark on the page you can do it like
% this:
%\IEEEpubid{0000--0000/00\$00.00~\copyright~2014 IEEE}
% or like this to get the Computer Society new two part style.
%\IEEEpubid{\makebox[\columnwidth]{\hfill 0000--0000/00/\$00.00~\copyright~2014 IEEE}%
%\hspace{\columnsep}\makebox[\columnwidth]{Published by the IEEE Computer Society\hfill}}
% Remember, if you use this you must call \IEEEpubidadjcol in the second
% column for its text to clear the IEEEpubid mark (Computer Society jorunal
% papers don't need this extra clearance.)



% use for special paper notices
%\IEEEspecialpapernotice{(Invited Paper)}



% for Computer Society papers, we must declare the abstract and index terms
% PRIOR to the title within the \IEEEtitleabstractindextext IEEEtran
% command as these need to go into the title area created by \maketitle.
% As a general rule, do not put math, special symbols or citations
% in the abstract or keywords.
\IEEEtitleabstractindextext{%
\begin{abstract}
Is it viable to use artificial intelligence, AI, to play Ms. Pac-Man? This journal will asses different AI algorithms used to play Ms. Pac-Man and evaluate their performance.

It is possible to implement AI that are as good as an average human. A Behaviour Tree was implemented and scored around the same as the author did making it a viable solution. The Artificial Neural Network performed worse but this seems to be a result of the evolution framework developed not being able to change the topology and time constraints in general. Monte Carlo Tree Search did not seem to be viable at first but after observing the behaviour of the AI, it is clear that one or more bugs reside in the code, why no conclusion on the performance of this approach can be made.
\end{abstract}

% Note that keywords are not normally used for peerreview papers.
\begin{IEEEkeywords}
Artificial Intelligence, Artificial Neural Network, Monte Carlo Tree Search, Behaviour Tree, Ms. Pac-Man.
\end{IEEEkeywords}}


% make the title area
\maketitle


% To allow for easy dual compilation without having to reenter the
% abstract/keywords data, the \IEEEtitleabstractindextext text will
% not be used in maketitle, but will appear (i.e., to be "transported")
% here as \IEEEdisplaynontitleabstractindextext when the compsoc 
% or transmag modes are not selected <OR> if conference mode is selected 
% - because all conference papers position the abstract like regular
% papers do.
\IEEEdisplaynontitleabstractindextext
% \IEEEdisplaynontitleabstractindextext has no effect when using
% compsoc or transmag under a non-conference mode.



% For peer review papers, you can put extra information on the cover
% page as needed:
% \ifCLASSOPTIONpeerreview
% \begin{center} \bfseries EDICS Category: 3-BBND \end{center}
% \fi
%
% For peerreview papers, this IEEEtran command inserts a page break and
% creates the second title. It will be ignored for other modes.
\IEEEpeerreviewmaketitle



\IEEEraisesectionheading{\section{Introduction}\label{sec:introduction}}
% Computer Society journal (but not conference!) papers do something unusual
% with the very first section heading (almost always called "Introduction").
% They place it ABOVE the main text! IEEEtran.cls does not automatically do
% this for you, but you can achieve this effect with the provided
% \IEEEraisesectionheading{} command. Note the need to keep any \label that
% is to refer to the section immediately after \section in the above as
% \IEEEraisesectionheading puts \section within a raised box.




% The very first letter is a 2 line initial drop letter followed
% by the rest of the first word in caps (small caps for compsoc).
% 
% form to use if the first word consists of a single letter:
% \IEEEPARstart{A}{demo} file is ....
% 
% form to use if you need the single drop letter followed by
% normal text (unknown if ever used by IEEE):
% \IEEEPARstart{A}{}demo file is ....
% 
% Some journals put the first two words in caps:
% \IEEEPARstart{T}{his demo} file is ....	
% 
% Here we have the typical use of a "T" for an initial drop letter
% and "HIS" in caps to complete the first word.
\IEEEPARstart{C}{an} a computer play Ms. Pac-Man? That is the question this journal seeks to answer. Multiple artificial intelligence, AI, algorithms will be analysed. They will have their performance assessed in regards to playing Ms. Pac-Man.
The report will evaluate Artificial Neural Networks, referred to as ANN, Monte Carlo Tree Search, referred to as MCTS and Behaviour Tree, referred to as BT.
% You must have at least 2 lines in the paragraph with the drop letter
% (should never be an issue)

% needed in second column of first page if using \IEEEpubid
%\IEEEpubidadjcol

% !TEX root = Report.tex

\section{Artificial Neural Network}
Artificial Neural Networks are inspired by our brain. It mimics the neurons and synapses of our brain. An ANN consists of layers of neurons that are connected to the adjacent layers of neurons. Not all neurons in one layer has to be connected to all other neurons in adjacent layers.

\subsection{Description}
In the ANN implemented in this journal, each layer of neurons are fully connected to every neuron in the adjacent layers. A connection between two neurons has a weight and a neuron has a bias. 

The first layer is the input neurons, and the last layer is the output neurons. In addition to the input and output layers, there is the optional hidden layer, which consists of any number of layers.

The output of input neurons is calculated as $ I_i + bias_i $, where $I_i$ is the input and $bias_i$ is the bias of the \textit{i'th} input neuron.

The output of neurons in the subsequent layers are calculated as $ \left( \displaystyle \sum_{i = 0}^{n} O_i * W_i \right) + bias $, where $O_i$ is the output from neuron $i$ in the previous layer, $W_i$ is the weight between neuron $i$ in the previous layer and the current neuron and $bias$ is the bias of the current neuron.

\subsection{Inputs to the Artificial Neural Network}
\label{subsub:ann_inputs}
The ANN developed for playing Ms. Pac-Man takes ten inputs: distances to all four ghosts, whether each ghost is edible, distance to nearest power pill and distance to the nearest pill.

\subsection{Outputs from the Artificial Neural Network}
\label{subsub:ann_outputs}
In this particular ANN there is only one output node. This node is interpreted as how good a move is.

Every time the Ms. Pac-Man framework requests a move, the list of possible moves are. For each possible move a run through the ANN is made, to determine how good that move is. The move with the highest output is then chosen as the move to do.

Another approach that could have been chosen, was to have four output nodes, each representing one of the four unique moves.

% !TEX root = Report.tex

% Diagram is created via the neuralnetwork environment
\subsection{Visual representation of the ANN}
\begin{neuralnetwork} [nodespacing=10mm, layerspacing=30mm,
		maintitleheight=2.5em, layertitleheight=10em,
		height=5, toprow=false, nodesize=17pt, style={},
		title={}, titlestyle={}]
	% nodespacing = vertical spacing between nodes
	% layerspacing = horizontal spacing between layers
	% maintitleheight = space reserved for main title
	% layertitleheight = space reserved for layer titles
	% height = max nodes in any one layer [REQUIRED]
	% toprow = top row in node space reserved for bias nodes
	% nodesize = size of nodes
	% style = style for internal "tikzpicture" environment
	% titlestyle = style for main title
	
	% To customize the text within nodes, define a macro like:
	%\newcommand{\mynodetext}[2] {
		% layer=#1, node=#2
	%}
	% The assign it:
	%\setdefaultnodetext{\mynodetext}
	
	% To draw a layer (advanced):
	%\layer[title={}, titlestyle={}, count=5,
	%		nodeclass={hidden neuron}, biaspos=top
	%		top=false, exclude={}, widetitle=false,
	%		bias={}]
	% title = layer title
	% titlestyle = style for layer title
	% count = nodes in this layer (excl. bias node)
	% nodeclass = class of node: 
	%		"neuron", which is base class for:
	%		"input neuron"
	%		"output neuron"
	%		"hidden neuron"
	% biaspos = position of bias node:
	%		"top" - immediately above other nodes
	%		"top row" - at the top of the node space, in
	%		the area reserved by setting parameter
	%		"toprow = true" on the "neuralnetwork"
	%		environment
	%		"center" - center of layer
	%		"center left" - left of "center"
	%		"center right" - right of "center"
	%		"center left left" - left of "center left"
	%		"center right right" - right of "center right"
	% top = layer is vertically aligned to the top, not the middle
	% exclude = one-based indices of nodes to avoid drawing
	% widetitle = allocate extra horizontal space for layer title
	% bias = true to add a bias node, false otherwise [REQUIRED]
		% To draw a layer (simpler):
	% "\layer" is wrapped up into some simpler macros, which can
	% accept the same parameters as "\layer"
	\inputlayer[title=Input layer,bias=false,count=10]	% bias=true, nodeclass={input neuron}
	\hiddenlayer[title=Hidden layer,bias=false,count=7]	% bias=true, nodeclass={hidden neuron}
	\linklayers
%	\hiddenlayer[count=12]	% bias=true, nodeclass={hidden neuron}
%	\linklayers
	\outputlayer[title=Output layer,count=1]	% bias=false, nodeclass={output neuron}
	\linklayers

	% "\linklayers" links every node in the previous layer to the
	% layer before it.

	% To customize the text on the links, define a macro like:
	%\newcommand{\mylinktext}[4] {
		% from layer=#1, from node=#2
		% to layer=#3, to node=#4
	%}
	% Then assign it:
	%\setdefaultlinklabel{\mylinktext}

	% To link the nodes of the previous two layers together:
	%\linklayers[title={}, style={}, not from={}, not to={}]
	% title = title to put above the links (i.e. between layers)
	% style = passed to "style" param of "\link"
	% not from = one-based indices of nodes to NOT link from
	% not to = one-based indices of nodes to NOT link to

	% To draw a link from one node to another:
	%\link[style={}, labelpos=midway, from layer, from node,
	%		to layer, to node]
	% style = style for the link path
	% labelpos = TikZ position of the label node along the path:
	%		"midway", "near end", "very near start",
	%		"at start", "pos=0.35", etc
	% from layer = layer to link from [REQUIRED]
	% from node = node to link from [REQUIRED]
	% to layer = layer to link to [REQUIRED]
	% to node = node to link to [REQUIRED]
\end{neuralnetwork}

\subsection{Teaching an Artificial Neural Network}
There are multiple ways to learn an ANN how to do best. The four major ways are: Supervised learning, unsupervised learning, reinforcement learning and evolutionary learning.

\subsubsection{Supervised learning}
Supervised learning requires example data, where the input and output is known. By minimising the difference between the actual output and the expected output you can learn the network what to do.

\subsubsection{Unsupervised learning}
Unsupervised learning does not require example data. A cost function is chosen and the goal is to minimise the cost.

\subsubsection{Reinforcement learning}
Instead of example data reinforcement learning observes someone or something else playing, using that as the reference.

\subsubsection{Evolutionary learning}
A completely different approach for teaching an ANN is to use an evolutionary approach. The idea comes from biology and the ongoing evolution of every species. By using evolution it is possible to evolve ANN that can solve tasks without learning it what to do, for instance by making small mutations to networks and evolving the best ones.

\subsection{Teaching an ANN to play Ms. Pac-Man}
The chosen teaching paradigm is evolution since it fits the jobs the most. There is no training data neither a cost function. Using reinforcement learning imposes an upper bound of ones own skill at playing Ms. Pac-Man.

\subsubsection{Algorithm parameters}
In order to teach the ANN to play Ms. Pac-Man an evolution framework was developed. Both the ANN and the evolution framework has multiple parameters to play with in order to achieve the best performing AI. \\

\noindent The important parameters include the following:

\begin{itemize}
	\item Network
	\begin{itemize}
		\item Number of input neurons
		\item Number of output neurons
		\item Number of neurons in the hidden layer
		\item Initial weight of connections
		\item Initial bias of neurons
	\end{itemize}
	
	\item Evolution
	\begin{itemize}
		\item Evaluations per network
		\item Networks in a generation
		\item Number of elitists
		\item Number of parents
		\item Chance of mutation
		\item Intensity of mutation
	\end{itemize}
\end{itemize}

\noindent Several runs of the evolution framework has been made with different kind of parameters in order to find the best performing network.

The number of input neurons defines how many inputs that are given to the network and the number of output neurons defines how many outputs the network will produce. The number of neurons in the hidden layer defines how many neurons there are in the hidden layer. The evolution framework does support multiple hidden layers, but is cumbersome to use at the moment. This decision was made to limit the amount of parameters and thereby limiting the search space.

Initial weight of connections is used to set the initial weight of the connections in the first network. Initial bias of neurons is the same as the aforementioned. A different approach could be to just randomise every weight and bias of the network. I chose the fixed value approach because it is slightly simpler to implement, and the value will change over time anyway.

Evaluations per network is how many games are played with a given network. Since Ms. Pac-Man is not deterministic, one can use a high number of evaluations to minimise the noise created from the indeterministic behaviour.

The number of elitists are the survivors of each generation. These are the networks that will be transferred to the next generation without being mutated.

The number of parents is the amount n best scoring networks that will be mutated in order to produce the remaining population.

Chance and intensity of mutation is how likely	 a given weight or neuron is to mutate, and how intense that mutation is.

\subsection{Experiments}
In order to develop the best network I ran the evolution framework several times adjusting different parameters each time. A small sample of these experiments are shown in the following table. For all experiments shown, the chance to mutate was 0.1 and the intensity of mutation was 0.2. The size of every generation was 100 and each network was evaluated 25 times.

% Column definition used to make fixed width right aligned columns
\newcolumntype{R}[1]{>{\raggedleft\arraybackslash}p{#1\columnwidth}}

\noindent \\
\begin{tabular}{ | r | r | r | r | }
	\hline
	Hidden neurons & Parents & Avg. score & Best avg. score \\ \hline
	0  & 2  & 2080 & 2632 \\ \hline
	0  & 10 & 2345 & 2868 \\ \hline
	7  & 2  & 1634 & 3458 \\ \hline
	7  & 10 & 2078 & 2799 \\ \hline
	15 & 2  & 2208 & 2696 \\ \hline
	15 & 10 & 2273 & 2635 \\
	\hline
\end{tabular}

\noindent \\
The best performing network i found uses ten inputs as described in section~\ref{subsub:ann_inputs} and one output as described in section~\ref{subsub:ann_outputs}. The number of neurons in the hidden layer are seven.
% !TEX root = Report.tex

\section{Monte Carlo Tree Search}
Monte Carlo Tree Search is the second algorithm explored in this journal. It is an algorithm that has proved its worth playing go. Since 2007 all the best AI agents for playing go has been implemented using MCTS\cite[slide~5]{lecture:mcts}. Go is a complex game so I figured that if it can do well on Go, it might be able to do well in Ms. Pac-Man as well.

\subsection{Description}
MCTS is a search algorithm that operates on trees.
The algorithm creates a new tree every time a request for a move is made. The first thing the algorithm does is to create a root node with the initial state. It will then repeat four steps as long as the algorithm is within some defined computational budget. The four steps it uses are: Selection, expansion, simulation and back-propagation.

\subsubsection{Selection}
Select successive children until a leaf node is reached, beginning from the root node. How the children are selected depends on implementation.

\subsubsection{Expansion}
Create one or more child nodes of the previously chosen node. Choose one of these to use for simulation.

\subsubsection{Simulation}
Do a random play-out of the game beginning from the state of the expanded node.

\subsubsection{Back-propagation}
Update information regarding score and visit count in the nodes on the path from the root node to the expanded node.

\subsection{Representation}
Each node in the tree has information regarding the parent, move, number of visits, total score, children and the game state.

The parent is used to find the root node. It is the only node with \texttt{null} as parent, since it has no parent.

The move determines what move was made to get to the given state of the node.

Number of visits are the amount of times the node has been visited, either directly or through one of its children og children's children etc.

Total score is the total score of the node itself plus all the children and the children's children etc.

Children are the children of the node. Each one representing one of the up to four uniques moves that are allowed.

Game state is the state of the game.

\subsection{Algorithm parameters}
Since the algorithm explore possible moves and evaluate how good a move is, there is not need for too many parameters. The only parameter that the algorithm uses, is a depth parameter. It defines how many moves the simulation step should simulate.

Having a high number of simulated moves per simulation means that there a simulated move is more likely to reflect the goodness of said move but at the cost of not being able to explore many different moves.

Having a smaller number of simulated moves gives more time to expand more nodes.

\subsection{Experiments}
\label{sub:mcts_experiments}
In order to find the best depth to run the algorithm with, i ran multiple experiments, where the only variable was the depth. Every depth was tested with 100 runs of the algorithm to try and minimise the noise produced as a result of the nondeterministic behaviour of the ghosts.

\noindent \\
\begin{tabular}{ | r | r | }
	\hline
	Depth & Score \\ \hline
	10  & 491.2 \\ \hline
	25  & 503.4 \\ \hline
	50  & 600.6 \\ \hline
	100 & 582.1 \\ \hline
	500 & 872.4 \\ \hline
	2500 & 1077.4 \\ \hline
	5000 & 878.2 \\ \hline
	10000 & 867.8 \\
	\hline
\end{tabular} \\

\noindent
Contrary to my expectations the depth needed was way higher than first anticipated.
I expected the optimal depth to be around 25-50, but the optimal seems to be around tenfold that. I believe the reason for this is that advancing the game state i extremely quick and therefore the depth is not a bottleneck before being way higher.

I ended up settling with a depth of 2500 moves per simulation since that was the best yielded the highest average score.
































% !TEX root = Report.tex

\section{Behaviour Tree}
Behaviour Trees are used to describe behaviour of non-player characters. Behaviour Trees are used in different games, for instance in EAs game Spore\cite{spore}.

\subsection{Description}
A BT is a directed tree. A node can either be the root node, a control flow node or a given task. The root node has no parent as the only node. Every node has one or more children, unless it is a task. Each node can be run and will return one of the following results:

\begin{itemize}
	\item Success
	\item Failure
	\item Running
\end{itemize} 

Success is returned when a task or control flow is successful. Failure is returned when a task or control flow failed. Running is returned when the given task or control flow node is still running.

\subsubsection{Control flow node}
A control flow node is a node that controls the flow in the tree. There are many types of control flow nodes. Some of the most common include: \\

\newcommand{\nodeHeading}[1]{\noindent \textbf{#1}\\}

\nodeHeading{Sequence}
Consists of multiple children. Will visit every child node in sequence beginning with the first. If it returns success it will visit the next child. If it returns failure it will return failure. If all children has been visited and they all returned success the sequence node will return success. \\

\nodeHeading{Selector}
Consists of multiple children. Will visit every child node in sequence beginning with the first. If it returns success it will return success. If it returns failure it will visits the next child. If all children has been visited and they all returned failure the selector node will return failure. \\

\nodeHeading{Inverter}
Consists of a single child. Will visit its child and return the opposite of what was returned. If success was returned it will return failure. If failure was returned it will return success. \\

\nodeHeading{Succeeder}
Consists of a single child. Will visit its child and return success regardless of what was returned by the children. The opposite of this node is not needed as an inverter with a succeeder as a child is exactly that. \\

\nodeHeading{Repeater}
Consists of a single child. Will visit its child and when either success or failure is returned from the child it will visit its child again. \\

\nodeHeading{Repeat Until Fail}
Consists of a single child. Will visit its child. If success is returned it will visit its child again. If failure is returned it will return failure to its parent.

\subsection{Algorithm parameters}
I have only used one parameter for this algorithm, flee distance, which is the distance at which Pac-Man will start fleeing, if a ghost is gloser than that.

\subsection{Implementation}
The implemented behaviour tree is shown below. Each node is an abbreviation of a node or task:

\begin{description}
	\item[R] Root node
	\item[Sq] Sequence control flow node
	\item[Sl] Selector control flow node
	\item[GTC] GhostTooClose task
	\item[PPCTC] PowerPillCloserThanGhost task
	\item[EPP] EatPowerPill task
	\item[RA] RunAway task
	\item[CEQ] CanEatGhost
	\item[EG] EatGhost
	\item[EP] EatPill
\end{description}

\noindent
\Tree[.R [.Sq 
				[.GTC ]
				[.Sl
					[.Sq 
						[.PPCTG ]
						[.EPP ]]
					[.RA ]]]
			[.Sq 
				[.CEG ]
				[.EG ]]
			[.EP ]]

\subsection{Experiments}
The only experiments done is regarding the flee distance. The game was run 1000 times with each distance and an average score was found. The ghost controller used is the \texttt{Legacy} controller. The results is presented below. \\

\begin{tabular}{ | r | l | }
	\hline
	Flee dist. & Avg. score \\ \hline
	4 & 5960.95 \\ \hline
	5 & 6279.24 \\ \hline
	6 & 6143.34 \\ \hline
	10 & 5630.81 \\ \hline
	20 & 3707.33 \\ \hline
	30 & 2780.95 \\ \hline
	40 & 2589.87 \\ \hline
	45 & 5256.68 \\ \hline
	50 & 4538.92 \\ \hline
	55 & 3684.87 \\ \hline
	60 & 2685.32 \\
	\hline
\end{tabular} \\

\noindent
The table shows there are a clear difference between the performance using different fleeing distances. The best yielding fleeing distance is five.

I find it peculiar as to why a fleeing distance of around 45-50 yields relatively high results compared to values around that. I have no reasonable explanation as why this happens. \\

\noindent
I ran another experiment using the \texttt{AggressiveGhosts} controller. Once again the game was run 1000 times per fleeing distance. \\

\begin{tabular}{ | r | l | }
	\hline
	Flee dist. & Avg. score \\ \hline
	4 & 6259.07 \\ \hline
	5 & 6463.39 \\ \hline
	6 & 6417.5 \\ \hline
	10 & 8317.4 \\ \hline
	20 & 8511.91 \\ \hline
	30 & 3207.33 \\ \hline
	40 & 2042.7 \\ \hline
	45 & 5792.69 \\ \hline
	50 & 3104.77 \\ \hline
	55 & 2822.72 \\ \hline
	60 & 1887.6 \\
	\hline
\end{tabular} \\

It is interesting to see how another flee distance is better when the ghost behave differently, i.e. another ghost controller is used. 



















\section{Conclusion}
The conclusion goes here.

% An example of a floating figure using the graphicx package.
% Note that \label must occur AFTER (or within) \caption.
% For figures, \caption should occur after the \includegraphics.
% Note that IEEEtran v1.7 and later has special internal code that
% is designed to preserve the operation of \label within \caption
% even when the captionsoff option is in effect. However, because
% of issues like this, it may be the safest practice to put all your
% \label just after \caption rather than within \caption{}.
%
% Reminder: the "draftcls" or "draftclsnofoot", not "draft", class
% option should be used if it is desired that the figures are to be
% displayed while in draft mode.
%
%\begin{figure}[!t]
%\centering
%\includegraphics[width=2.5in]{myfigure}
% where an .eps filename suffix will be assumed under latex, 
% and a .pdf suffix will be assumed for pdflatex; or what has been declared
% via \DeclareGraphicsExtensions.
%\caption{Simulation results for the network.}
%\label{fig_sim}
%\end{figure}

% Note that IEEE typically puts floats only at the top, even when this
% results in a large percentage of a column being occupied by floats.
% However, the Computer Society has been known to put floats at the bottom.


% An example of a double column floating figure using two subfigures.
% (The subfig.sty package must be loaded for this to work.)
% The subfigure \label commands are set within each subfloat command,
% and the \label for the overall figure must come after \caption.
% \hfil is used as a separator to get equal spacing.
% Watch out that the combined width of all the subfigures on a 
% line do not exceed the text width or a line break will occur.
%
%\begin{figure*}[!t]
%\centering
%\subfloat[Case I]{\includegraphics[width=2.5in]{box}%
%\label{fig_first_case}}
%\hfil
%\subfloat[Case II]{\includegraphics[width=2.5in]{box}%
%\label{fig_second_case}}
%\caption{Simulation results for the network.}
%\label{fig_sim}
%\end{figure*}
%
% Note that often IEEE papers with subfigures do not employ subfigure
% captions (using the optional argument to \subfloat[]), but instead will
% reference/describe all of them (a), (b), etc., within the main caption.
% Be aware that for subfig.sty to generate the (a), (b), etc., subfigure
% labels, the optional argument to \subfloat must be present. If a
% subcaption is not desired, just leave its contents blank,
% e.g., \subfloat[].


% An example of a floating table. Note that, for IEEE style tables, the
% \caption command should come BEFORE the table and, given that table
% captions serve much like titles, are usually capitalized except for words
% such as a, an, and, as, at, but, by, for, in, nor, of, on, or, the, to
% and up, which are usually not capitalized unless they are the first or
% last word of the caption. Table text will default to \footnotesize as
% IEEE normally uses this smaller font for tables.
% The \label must come after \caption as always.
%
%\begin{table}[!t]
%% increase table row spacing, adjust to taste
%\renewcommand{\arraystretch}{1.3}
% if using array.sty, it might be a good idea to tweak the value of
% \extrarowheight as needed to properly center the text within the cells
%\caption{An Example of a Table}
%\label{table_example}
%\centering
%% Some packages, such as MDW tools, offer better commands for making tables
%% than the plain LaTeX2e tabular which is used here.
%\begin{tabular}{|c||c|}
%\hline
%One & Two\\
%\hline
%Three & Four\\
%\hline
%\end{tabular}
%\end{table}


% Note that the IEEE does not put floats in the very first column
% - or typically anywhere on the first page for that matter. Also,
% in-text middle ("here") positioning is typically not used, but it
% is allowed and encouraged for Computer Society conferences (but
% not Computer Society journals). Most IEEE journals/conferences use
% top floats exclusively. 
% Note that, LaTeX2e, unlike IEEE journals/conferences, places
% footnotes above bottom floats. This can be corrected via the
% \fnbelowfloat command of the stfloats package.

% Can use something like this to put references on a page
% by themselves when using endfloat and the captionsoff option.
\ifCLASSOPTIONcaptionsoff
  \newpage
\fi



% can use a bibliography generated by BibTeX as a .bbl file
% BibTeX documentation can be easily obtained at:
% http://www.ctan.org/tex-archive/biblio/bibtex/contrib/doc/
% The IEEEtran BibTeX style support page is at:
% http://www.michaelshell.org/tex/ieeetran/bibtex/
%\bibliographystyle{IEEEtran}
% argument is your BibTeX string definitions and bibliography database(s)
%\bibliography{IEEEabrv,../bib/paper}
%
% <OR> manually copy in the resultant .bbl file
% set second argument of \begin to the number of references
% (used to reserve space for the reference number labels box)
\begin{thebibliography}{1}

\bibitem{lecture:mcts}
S.~Risi, lecture notes on MCTS, \url{https://learnit.itu.dk/pluginfile.php/140291/mod_resource/content/1/8mcts.pdf}

\bibitem{spore}
C.~Checker, My Liner Notes for Spore/Spore Behavior Tree Docs, \url{http://chrishecker.com/My_liner_notes_for_spore/Spore_Behavior_Tree_Docs}

%\bibitem{IEEEhowto:kopka}
%H.~Kopka and P.~W. Daly, \emph{A Guide to \LaTeX}, 3rd~ed.\hskip 1em plus 0.5em minus 0.4em\relax Harlow, England: Addison-Wesley, 1999.

\end{thebibliography}

% biography section
% 
% If you have an EPS/PDF photo (graphicx package needed) extra braces are
% needed around the contents of the optional argument to biography to prevent
% the LaTeX parser from getting confused when it sees the complicated
% \includegraphics command within an optional argument. (You could create
% your own custom macro containing the \includegraphics command to make things
% simpler here.)
%\begin{IEEEbiography}[{\includegraphics[width=1in,height=1.25in,clip,keepaspectratio]{mshell}}]{Michael Shell}


% insert where needed to balance the two columns on the last page with
% biographies
%\newpage

% You can push biographies down or up by placing
% a \vfill before or after them. The appropriate
% use of \vfill depends on what kind of text is
% on the last page and whether or not the columns
% are being equalized.

%\vfill

% Can be used to pull up biographies so that the bottom of the last one
% is flush with the other column.
%\enlargethispage{-5in}



% that's all folks
\end{document}