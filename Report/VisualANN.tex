% !TEX root = Report.tex

% Diagram is created via the neuralnetwork environment
\subsection{Visual representation of the ANN}
\begin{neuralnetwork} [nodespacing=10mm, layerspacing=30mm,
		maintitleheight=2.5em, layertitleheight=10em,
		height=5, toprow=false, nodesize=17pt, style={},
		title={}, titlestyle={}]
	% nodespacing = vertical spacing between nodes
	% layerspacing = horizontal spacing between layers
	% maintitleheight = space reserved for main title
	% layertitleheight = space reserved for layer titles
	% height = max nodes in any one layer [REQUIRED]
	% toprow = top row in node space reserved for bias nodes
	% nodesize = size of nodes
	% style = style for internal "tikzpicture" environment
	% titlestyle = style for main title
	
	% To customize the text within nodes, define a macro like:
	%\newcommand{\mynodetext}[2] {
		% layer=#1, node=#2
	%}
	% The assign it:
	%\setdefaultnodetext{\mynodetext}
	
	% To draw a layer (advanced):
	%\layer[title={}, titlestyle={}, count=5,
	%		nodeclass={hidden neuron}, biaspos=top
	%		top=false, exclude={}, widetitle=false,
	%		bias={}]
	% title = layer title
	% titlestyle = style for layer title
	% count = nodes in this layer (excl. bias node)
	% nodeclass = class of node: 
	%		"neuron", which is base class for:
	%		"input neuron"
	%		"output neuron"
	%		"hidden neuron"
	% biaspos = position of bias node:
	%		"top" - immediately above other nodes
	%		"top row" - at the top of the node space, in
	%		the area reserved by setting parameter
	%		"toprow = true" on the "neuralnetwork"
	%		environment
	%		"center" - center of layer
	%		"center left" - left of "center"
	%		"center right" - right of "center"
	%		"center left left" - left of "center left"
	%		"center right right" - right of "center right"
	% top = layer is vertically aligned to the top, not the middle
	% exclude = one-based indices of nodes to avoid drawing
	% widetitle = allocate extra horizontal space for layer title
	% bias = true to add a bias node, false otherwise [REQUIRED]
		% To draw a layer (simpler):
	% "\layer" is wrapped up into some simpler macros, which can
	% accept the same parameters as "\layer"
	\inputlayer[title=Input layer,bias=false,count=10]	% bias=true, nodeclass={input neuron}
	\hiddenlayer[title=Hidden layer,bias=false,count=7]	% bias=true, nodeclass={hidden neuron}
	\linklayers
%	\hiddenlayer[count=12]	% bias=true, nodeclass={hidden neuron}
%	\linklayers
	\outputlayer[title=Output layer,count=1]	% bias=false, nodeclass={output neuron}
	\linklayers

	% "\linklayers" links every node in the previous layer to the
	% layer before it.

	% To customize the text on the links, define a macro like:
	%\newcommand{\mylinktext}[4] {
		% from layer=#1, from node=#2
		% to layer=#3, to node=#4
	%}
	% Then assign it:
	%\setdefaultlinklabel{\mylinktext}

	% To link the nodes of the previous two layers together:
	%\linklayers[title={}, style={}, not from={}, not to={}]
	% title = title to put above the links (i.e. between layers)
	% style = passed to "style" param of "\link"
	% not from = one-based indices of nodes to NOT link from
	% not to = one-based indices of nodes to NOT link to

	% To draw a link from one node to another:
	%\link[style={}, labelpos=midway, from layer, from node,
	%		to layer, to node]
	% style = style for the link path
	% labelpos = TikZ position of the label node along the path:
	%		"midway", "near end", "very near start",
	%		"at start", "pos=0.35", etc
	% from layer = layer to link from [REQUIRED]
	% from node = node to link from [REQUIRED]
	% to layer = layer to link to [REQUIRED]
	% to node = node to link to [REQUIRED]
\end{neuralnetwork}